\section{Variable Length Code}
\begin{frame}{Can we do more better?}
    \begin{columns}
    \column{.7\textwidth}
\begin{table}[]
\begin{tabular}{ccc}
\rowcolor[HTML]{009901} 
Character                  & Frequency                 & Encoding \\ 
\rowcolor[HTML]{FE0000} 
A                          & 9                         & 100      \\
\rowcolor[HTML]{F8A102} 
B                          & 2                         & \alert{0} 11      \\
\rowcolor[HTML]{F8FF00} 
C                          & 1                         & \alert{0} 10      \\
\rowcolor[HTML]{FCFF2F} 
D                          & 1                         & \alert{00} 1      \\
\rowcolor[HTML]{FFCC67} 
R                          & 2                         & \alert{00} 0      \\
\multicolumn{2}{c}{\cellcolor[HTML]{FFFFFF}Total = 15} &         
\end{tabular}
\end{table}
    \column{.3\textwidth}
    \begin{exampleblock}{\centering Intuition - 1}
    \centering
       Aren't leading zeroes \alert{redundant}?
    \end{exampleblock}
    \end{columns}

\end{frame}
\begin{frame}{Variable length encoding}
    \begin{center}
         \begin{columns}
    \column{.7\textwidth}
    % Please add the following required packages to your document preamble:
% \usepackage[table,xcdraw]{xcolor}
% If you use beamer only pass "xcolor=table" option, i.e. \documentclass[xcolor=table]{beamer}
\begin{table}[]
\begin{tabular}{ccc}
\rowcolor[HTML]{009901} 
Character                  & Frequency                 & Encoding \\
\rowcolor[HTML]{FE0000} 
A                          & 9                         & 100      \\
\rowcolor[HTML]{F8A102} 
B                          & 2                         & 11       \\
\rowcolor[HTML]{F8FF00} 
C                          & 1                         & 10       \\
\rowcolor[HTML]{FCFF2F} 
D                          & 1                         & 1        \\
\rowcolor[HTML]{FFCC67} 
R                          & 2                         & 0        \\
\multicolumn{2}{c}{\cellcolor[HTML]{FFFFFF}Total = 15} &         
\end{tabular}
\end{table}
    \pause 
    \column{.3\textwidth}
    \begin{alertblock}{Trust me!}
    \end{alertblock}
    
    \pause
    \begin{exampleblock}{\centering bits needed}
    \centering
       $3  \times 9 + 2 \times 2 + 2 \times 1  + 1\times 1  + 1 \times 2  = 36$
    \end{exampleblock}
    \end{columns}
    \end{center}
% \end{frame}
\end{frame}
\begin{frame}{Can we do better ? }
    % \pause
    \begin{block}{\centering Observation - 2}
    \centering 
        Why are we not using the frequency of the characters ? 
    \end{block}
    \pause
    \begin{exampleblock}{\centering Intuition - 2}
    \centering 
    Characters with highest frequency should have less number of bits
    \end{exampleblock}
\end{frame}
\begin{frame}{Frequency based variable length encoding}
% \framesubtitle{the string}
    \begin{center}
         \begin{columns}
    \column{.7\textwidth}
    % Please add the following required packages to your document preamble:
% \usepackage[table,xcdraw]{xcolor}
% If you use beamer only pass "xcolor=table" option, i.e. \documentclass[xcolor=table]{beamer}
\begin{table}[]
\begin{tabular}{ccc}
\rowcolor[HTML]{009901} 
Character                  & Frequency                 & Encoding \pause\\
\rowcolor[HTML]{FE0000} 
A                          & 9                         & 0        \pause\\
\rowcolor[HTML]{F8A102} 
B                          & 2                         & 1        \pause\\
\rowcolor[HTML]{FFCC67} 
R                          & 2                         & 10       \pause\\
\rowcolor[HTML]{FCFF2F} 
C                          & 1                         & 11       \pause\\
\rowcolor[HTML]{F8FF00} 
D                          & 1                         & 100      \pause\\
\multicolumn{2}{c}{\cellcolor[HTML]{FFFFFF}Total = 15} &         
\end{tabular}
\end{table}
    \pause 
    \column{.3\textwidth}
    \begin{alertblock}{Trust me!}
    \end{alertblock}
    
    % \pause
    \begin{exampleblock}{\centering bits needed}
    \centering
       $1 \times 9 + 1 \times 2 + 2 \times 1  + 3\times 1  + 2 \times 2  = 20$
    \end{exampleblock}
    \end{columns}
    \end{center} 
\end{frame}
\begin{frame}{Takeaway - 2}
\begin{block}{\centering Takeaway - 2}
\centering
    Frequency based encoding is a good technique in reducing bits count.
\end{block}
\begin{table}[]
    \centering
    \begin{tabular}{|c|c|c|c|}
    \hline
        ASCII  & Mod. F.L. &  V. L.(Rand.) & V. L.(Frequency) \\\hline
         120 & 45 & 36 & 20 \\ \hline 
    \end{tabular}
    \caption{Bits needed in different encoding}
    % \label{tab:my_label}
\end{table}    
\end{frame}


\begin{frame}{But...}
\framesubtitle{Is our encoding right?}
\only<1-2>{Is our approach right ?} 
    \only<2-2>{\\[2cm]Can it be decoded \alert{correctly}?}
    \pause \pause
    \begin{columns}
    \column{.5\textwidth}
    
    \begin{block}{\centering Encoded bits}
        \centering
        10
    \end{block}
    \\[1cm]
    \only<5->{
    \begin{block}{\centering Decoded Text - 1}
        \centering
        \only<5->{BA}
    \end{block}
    }
    \only<6->{
    \begin{block}{\centering Decoded Text - 2}
        \centering
        \only<6->{R}
    \end{block}
    }
   \column{.5\textwidth}
   % Please add the following required packages to your document preamble:
% \usepackage[table,xcdraw]{xcolor}
% If you use beamer only pass "xcolor=table" option, i.e. \documentclass[xcolor=table]{beamer}
\only<4->{
\begin{table}[]
\begin{tabular}{cc}
\rowcolor[HTML]{009901} 
Character & Encoding \\
\rowcolor[HTML]{FE0000} 
A         & 0        \\
\rowcolor[HTML]{F8A102} 
B         & 1        \\
\rowcolor[HTML]{FFCC67} 
R         & 10       \\
\rowcolor[HTML]{FCFF2F} 
C         & 11       \\
\rowcolor[HTML]{F8FF00} 
D         & 100     
\end{tabular}
% \caption{Our variable length encoding}
\end{table}
}
    \end{columns}
   
\end{frame}